% Options for packages loaded elsewhere
\PassOptionsToPackage{unicode}{hyperref}
\PassOptionsToPackage{hyphens}{url}
%
\documentclass[
]{book}
\usepackage{amsmath,amssymb}
\usepackage{lmodern}
\usepackage{iftex}
\ifPDFTeX
  \usepackage[T1]{fontenc}
  \usepackage[utf8]{inputenc}
  \usepackage{textcomp} % provide euro and other symbols
\else % if luatex or xetex
  \usepackage{unicode-math}
  \defaultfontfeatures{Scale=MatchLowercase}
  \defaultfontfeatures[\rmfamily]{Ligatures=TeX,Scale=1}
\fi
% Use upquote if available, for straight quotes in verbatim environments
\IfFileExists{upquote.sty}{\usepackage{upquote}}{}
\IfFileExists{microtype.sty}{% use microtype if available
  \usepackage[]{microtype}
  \UseMicrotypeSet[protrusion]{basicmath} % disable protrusion for tt fonts
}{}
\makeatletter
\@ifundefined{KOMAClassName}{% if non-KOMA class
  \IfFileExists{parskip.sty}{%
    \usepackage{parskip}
  }{% else
    \setlength{\parindent}{0pt}
    \setlength{\parskip}{6pt plus 2pt minus 1pt}}
}{% if KOMA class
  \KOMAoptions{parskip=half}}
\makeatother
\usepackage{xcolor}
\usepackage{graphicx}
\makeatletter
\def\maxwidth{\ifdim\Gin@nat@width>\linewidth\linewidth\else\Gin@nat@width\fi}
\def\maxheight{\ifdim\Gin@nat@height>\textheight\textheight\else\Gin@nat@height\fi}
\makeatother
% Scale images if necessary, so that they will not overflow the page
% margins by default, and it is still possible to overwrite the defaults
% using explicit options in \includegraphics[width, height, ...]{}
\setkeys{Gin}{width=\maxwidth,height=\maxheight,keepaspectratio}
% Set default figure placement to htbp
\makeatletter
\def\fps@figure{htbp}
\makeatother
\setlength{\emergencystretch}{3em} % prevent overfull lines
\providecommand{\tightlist}{%
  \setlength{\itemsep}{0pt}\setlength{\parskip}{0pt}}
\setcounter{secnumdepth}{-\maxdimen} % remove section numbering
\newlength{\cslhangindent}
\setlength{\cslhangindent}{1.5em}
\newlength{\csllabelwidth}
\setlength{\csllabelwidth}{3em}
\newlength{\cslentryspacingunit} % times entry-spacing
\setlength{\cslentryspacingunit}{\parskip}
\newenvironment{CSLReferences}[2] % #1 hanging-ident, #2 entry spacing
 {% don't indent paragraphs
  \setlength{\parindent}{0pt}
  % turn on hanging indent if param 1 is 1
  \ifodd #1
  \let\oldpar\par
  \def\par{\hangindent=\cslhangindent\oldpar}
  \fi
  % set entry spacing
  \setlength{\parskip}{#2\cslentryspacingunit}
 }%
 {}
\usepackage{calc}
\newcommand{\CSLBlock}[1]{#1\hfill\break}
\newcommand{\CSLLeftMargin}[1]{\parbox[t]{\csllabelwidth}{#1}}
\newcommand{\CSLRightInline}[1]{\parbox[t]{\linewidth - \csllabelwidth}{#1}\break}
\newcommand{\CSLIndent}[1]{\hspace{\cslhangindent}#1}
\usepackage[]{graphicx}
\usepackage[]{color}
\usepackage{alltt}
\usepackage{caption}
\captionsetup[figure]{font=tiny}

\newcommand{\mytitle}{My Super Fancy Thesis Title}
\newcommand{\myname}{Jane Doe}
\newcommand{\mysupervisor}{Dr Dre}

\usepackage[a4paper, width = 160mm, top = 35mm, bottom = 30mm, 
            bindingoffset = 0mm]{geometry}
\usepackage[utf8]{inputenc}
\usepackage{ragged2e}
\usepackage{xcolor}
\usepackage[round, comma]{natbib}
\usepackage{fancyhdr}
\newcommand{\changefont}{%
  \fontsize{8}{11}\selectfont
}
\usepackage{hyperref}
\hypersetup{
  colorlinks = true,
  linkcolor = black,
  urlcolor = black,
  citecolor = black}
\pagestyle{fancy}
\fancyhead{}
\fancyhead[R]{\changefont{\mytitle}}
\fancyfoot{}
\fancyfoot[R]{\thepage}
\setlength{\headheight}{14.5pt}
\setlength{\parindent}{0pt}
\interfootnotelinepenalty = 10000

% ------------------------------------------------------------------------------
  % MAIN -------------------------------------------------------------------------
  % ------------------------------------------------------------------------------
  \IfFileExists{upquote.sty}{\usepackage{upquote}}{}
\begin{document}

% FRONT PAGE -------------------------------------------------------------------
\ifLuaTeX
  \usepackage{selnolig}  % disable illegal ligatures
\fi
\IfFileExists{bookmark.sty}{\usepackage{bookmark}}{\usepackage{hyperref}}
\IfFileExists{xurl.sty}{\usepackage{xurl}}{} % add URL line breaks if available
\urlstyle{same} % disable monospaced font for URLs
\hypersetup{
  pdftitle={Predicting Droughts in the Amazon Basin based on Global Sea Surface Temperatures},
  pdfauthor={Dario Lepke},
  hidelinks,
  pdfcreator={LaTeX via pandoc}}

\title{Predicting Droughts in the Amazon Basin based on Global Sea
Surface Temperatures}
\author{Dario Lepke}
\date{}

\begin{document}
\frontmatter
\maketitle

\mainmatter

% FRONT PAGE -------------------------------------------------------------------
  
\begin{titlepage}
\begin{center}

\LARGE
Master's Thesis
    
\vspace{0.5cm}
      
\rule{\textwidth}{1.5pt}
\LARGE
\textbf{Test}
\rule{\textwidth}{1.5pt}
   
\vspace{0.5cm}
      
\large
Department of Statistics \\
Ludwig-Maximilians-Universität München 

\vfill

\Large
\textbf{test}

\vfill

\large
Munich, Month Day\textsuperscript{th}, Year
      
\vfill

\includegraphics[width = 0.4\textwidth]{sigillum.png}

\vfill

\normalsize
Submitted in partial fulfillment of the requirements for the degree of M. Sc.
\\

Supervised by test

\end{center}
\end{titlepage}
\end{document}

\hypertarget{introduction}{%
\chapter*{Introduction}\label{introduction}}
\addcontentsline{toc}{chapter}{Introduction}

With future climate change droughts in the Amazon forest may become more
frequent and/or severe. Droughts can turn Amazon regions from rain
forest into savanna, leading to high amounts of carbon released into the
atmosphere. Therefore, predicting future droughts and understanding the
underlying mechanisms is of great interest. Ciemer et al.
(\protect\hyperlink{ref-ciemer2020early}{2020}), established an early
warning indicator for droughts in the central Amazon basin (CAB), based
on tropical Atlantic sea surface temperatures (SSTs). Inspired by their
work, the aim of this thesis is to build up on this work and improve its
predictive power by using different statistical methods. Here we seek to
build a model that is able to predict monthly precipitation based on the
sea surface temperatures. Also we want to identify those sea regions
that are most important for doing so, making interpretability a point of
interest, too. Firstly we will analyze the data descriptively to explore
patterns and spatial dependencies. This includes a cluster analysis of
the precipitation data in the central Amazon basin. Following we will
compare two different regression approaches and their capability to
predict precipitation as well as their interpretability of the SST
regions selected by them. The first model is the lasso as proposed by
Tibshirani (\protect\hyperlink{ref-tibshirani1996regression}{1996}).
Comparing different model specifications we will carry on the findings
from the lasso and fit a (sparse) fused lasso on the data (Tibshirani et
al. (\protect\hyperlink{ref-tibshirani2005sparsity}{2005})). Both models
will be evaluated using a 5-fold forward selection, a model evaluation
technique that takes into account the time dependencies present in the
data at hand. We conclude with a summary of the findings in this work
and give an overview of strengths and limitations of the approaches used
together with ideas for future research.

This thesis was written and supervised in cooperation with Dr.~Niklas
Boers from the Potsdam Institute for Climate Impact Research (Climate
Impact Research (PIK) e. V. (\protect\hyperlink{ref-PIC}{2022})) and
Dr.~Fabian Scheipl (LMU)

\hypertarget{refs}{}
\begin{CSLReferences}{1}{0}
\leavevmode\vadjust pre{\hypertarget{ref-ciemer2020early}{}}%
Ciemer, Catrin, Lars Rehm, Juergen Kurths, Reik V Donner, Ricarda
Winkelmann, and Niklas Boers. 2020. {``An Early-Warning Indicator for
Amazon Droughts Exclusively Based on Tropical Atlantic Sea Surface
Temperatures.''} \emph{Environmental Research Letters} 15 (9): 094087.

\leavevmode\vadjust pre{\hypertarget{ref-PIC}{}}%
Climate Impact Research (PIK) e. V., Potsdam Institute for. 2022.
{``Potsdam Insitute for Climate Impact Research.''} 2022.
\url{https://www.pik-potsdam.de/en}.

\leavevmode\vadjust pre{\hypertarget{ref-tibshirani1996regression}{}}%
Tibshirani, Robert. 1996. {``Regression Shrinkage and Selection via the
Lasso.''} \emph{Journal of the Royal Statistical Society: Series B
(Methodological)} 58 (1): 267--88.

\leavevmode\vadjust pre{\hypertarget{ref-tibshirani2005sparsity}{}}%
Tibshirani, Robert, Michael Saunders, Saharon Rosset, Ji Zhu, and Keith
Knight. 2005. {``Sparsity and Smoothness via the Fused Lasso.''}
\emph{Journal of the Royal Statistical Society: Series B (Statistical
Methodology)} 67 (1): 91--108.

\end{CSLReferences}

\backmatter
\end{document}
